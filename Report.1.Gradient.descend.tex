\documentclass[hidelinks]{report}
\usepackage{graphicx} % Required for inserting images
\usepackage{amsmath}
\usepackage{siunitx}
\usepackage{placeins}
\usepackage{float}
\usepackage{hyperref}
\usepackage{cite}
\usepackage[utf8]{inputenc}
\usepackage[T1]{fontenc}
\usepackage{xcolor,graphicx}
\setcounter{secnumdepth}{0}
\usepackage{titlesec}
\usepackage[top=1.4in,bottom=1.4in,right=1.25in,left=1.25in]{geometry}
\usepackage{rotating}
\usepackage{subcaption}
\usepackage{lipsum}
\usepackage{fancyhdr}
\usepackage{listing}

\definecolor{blue}{RGB}{31,56,100}

\usepackage{lipsum}% http://ctan.org/pkg/lipsum
\makeatletter
\def\@makechapterhead#1{%
  {
  \parindent \z@ \raggedright \normalfont
    
    \ifnum \c@secnumdepth >\m@ne
        \huge\bfseries \thechapter.\
    \fi
    \interlinepenalty\@M
    #1\par\nobreak%
    \vskip 40\p@% 
  }}
\makeatother

\renewcommand{\thesection}{\arabic{chapter}.\arabic{section}}
\renewcommand{\thesubsection}{\thesection.\arabic{subsection}}

\newcounter{subsecindex}[section]
\renewcommand{\thesubsecindex}{\thesubsection%
  \ifnum\value{subsecindex}>0
    .\arabic{subsecindex}%
  \fi
}

% Redefine the \section command to include the index
\let\oldsection\section
\renewcommand{\section}[1]{%
  \setcounter{subsecindex}{0} % Reset subsection counter for each section
  \refstepcounter{section}%
  \oldsection{\thesection\hspace{0.5em}#1}%
}

% Redefine the \subsection command to include the index
\let\oldsubsection\subsection
\renewcommand{\subsection}[1]{%
  \refstepcounter{subsection}%
  \oldsubsection{\thesubsecindex\hspace{0.5em}#1}%
}


\begin{document}
\begin{titlepage}
\begin{center}
\begin{figure}
    \centering
    \includegraphics[width=0.9\linewidth]{image/usth.png}
\end{figure}

\textsc{\Large }\\[1.5cm]

{\huge \bfseries \uppercase{Labwork 1} \\[3cm] }

% Title
\rule{\linewidth}{0.3mm} \\[0.4cm]
{ \huge \bfseries\color{blue} Gradient descend}
\rule{\linewidth}{0.3mm} \\[0.5cm]
\Large{Student:} 2440056 - Luong Thi Ngoc Diep\\[0.3cm]
\large Academic Year: 2024-2026
    
\end{center}
\end{titlepage}

\clearpage
\pagestyle{plain} 
\setcounter{page}{1}

\chapter{How I Implemented the Algorithm}

In this lab, I implemented Gradient Descent from scratch in Python without using any external libraries.

We aim to minimize the function:
\[
f(x) = x^2, \quad \text{with derivative} \quad f'(x) = 2x
\]

We start with an initial value of \( x \) and iteratively apply the update rule:
\[
x_{t+1} = x_t - r \cdot f'(x_t)
\]
Where:
\begin{itemize}
  \item \( r \) is the learning rate
  \item \( f'(x_t) = 2x_t \) is the gradient at current \( x \)
\end{itemize}

\begin{verbatim}
def gradient_descent(x, lr):
    threshold = 1
    loss = []
    xs = []
    count = 0  
    print("Iter\t\tx\t\tf(x)")
    print("-" * 50)

    while x > threshold:
        f = x**2
        print(f"{count}\t\t{x:.6f}\t\t{f:.6f}")
        loss.append(f)
        xs.append(x)

        f_prime = 2 * x
        x = x - lr * f_prime  
        count += 1

    return loss

loss = gradient_descent(10, 0.1)

\end{verbatim}

\chapter{Intermediate Results for $r = 0.1$}

\begin{center}
\begin{tabular}{|c|c|c|}
\hline
Iteration & $x$ & $f(x)$ \\
\hline
0 & 10.0000 & 100.0000 \\
1 & 8.0000 & 64.0000 \\
2 & 6.4000 & 40.9600 \\
3 & 5.1200 & 26.2144 \\
4 & 4.0960 & 16.7772 \\
5 & 3.2768 & 10.7374 \\
6 & 2.6214 & 6.8719 \\
7 & 2.0972 & 4.4020 \\
8 & 1.6777 & 2.8133 \\
9 & 1.3422 & 1.8005 \\
\hline
\end{tabular}
\end{center}

Graphical Visualization

\begin{figure}[H]
    \centering
    \includegraphics[width=0.5\linewidth]{image/loss_curve.png}
    \caption{Loss value $f(x)$ over iterations with $r = 0.1$}
\end{figure}

We tested multiple values of learning rate \( r \) and observed:

\begin{itemize}
    \item \textbf{Small \( r = 0.01 \):}  
    Converges slowly. Requires many iterations to reach a minimum.
\begin{figure}[H]
    \centering
    \includegraphics[width=0.5\linewidth]{001.png}
    \label{fig:enter-label}
\end{figure}
    \item \textbf{Optimal \( r = 0.1 \):}  
    Good balance of speed and stability. Smooth convergence to the minimum.

    \item \textbf{Large \( r = 0.5 \):}  
    Faster convergence initially, but may overshoot and oscillate around the minimum.
\begin{figure}[H]
    \centering
    \includegraphics[width=0.5\linewidth]{image/05.png}
\end{figure}
    \item \textbf{Too Large \( r = 1.0 \):}  
    Causes divergence or continuous bouncing around the minimum. The value of \( x \) may alternate and never settle.
    
\end{itemize}
In this lab, we implemented gradient descent to minimize \( f(x) = x^2 \). We found that a moderate learning rate (e.g., 0.1) offers a good trade-off between speed and stability.
\end{document}

